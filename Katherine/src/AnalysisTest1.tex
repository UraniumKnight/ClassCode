\documentclass{article}
\begin{document}
Katherine Beine \\
1.  (a)  Let $q_{1},...,q_{n},...$ be a list of distinct rational numbers. Consider  $\bigcap_{n=1}^{n=\infty} A_{n} = 0$.   Based on the proof for part b, \textbf{Q} - $\lbrace q_{n} \rbrace = 0$ does not create a contradiction because both \textbf{Q} and all $\lbrace q_{n} \rbrace$ are countable.  So there is no contradiction when the infinite intersection of all the countable stuff is empty.  \\
(b) In the reals, $x_{1},...,x_{n},...$ is a countable list of distinct real numbers, and the collection of all $x_{n}$ is countable.  If some $B_{n} \neq 0$, then any b $\in B_{n}$ will be in any $B_{k}$ where k $\leq$ n.  Thus, if $\bigcap_{n=1}^{n=\infty} B_{n}$ = 0, some $B_{k} = 0$.  So suppose $B_{k} = 0$.  Then $\textbf{R} - \lbrace x_{k} \rbrace = 0$, so $ \lbrace x_{k} \rbrace = \textbf{R}$.  However, \textbf{R} is uncountable and all $\lbrace x_{n} \rbrace$ are countable, so this is a contradiction.  Therefore $\bigcap_{n=1}^{n=\infty} B_{n} \neq 0$.  \\
\\
2.  Let A be obstructive in \textbf{R}.  Let x $\in$ \textbf{R}, and let $N_{r}(x)$ be a neighborhood of x with r $>$ 0.  Consider y $\in N_{r}(x), y \neq x$.  Then because A is obstructive in \textbf{R}, there is some a $\in$ A such that x$<$a$<$y.  Then a $\in N_{r}(x)$ and $a \neq x$ so x is a limit point of A.  Thus A is thick in \textbf{R}.\\
Now let A be thick in \textbf{R}.  Let x, y $\in$ \textbf{R} be given.  WLOG, let x $<$ y.  Consider $N_{r}(x)$ where r $=$ d(x, y).  Because x is a limit point of A, there is some point, $a \in N_{r}(x)$ such that $a \neq x, a \in A$.  Thus x$<$a$<$y.  Thus A is obstructive in X.  QED\\
\\
3.  Attempt to establish an equivalence relation from J to B where $b_{N} \in B$\\ \\if $b_{N} =\sum$\\
\\Then for N $\geq$ 2, $b_{N} = a_{1} \times b_{1} + a_{2} \times (-\sqrt{b_{2}}) + ...$ and $b_{N} = a_{1} \times b_{1} + a_{2} \times (\sqrt{b_{2}}) + ...$.  Thus J $\rightarrow$ B is not a 1-1 function, so B is not countable.  QED\\
\\
4.  Let M be a nonempty set bounded in \textbf{$R^{k}$}, and let $\delta > 0$ be given.  Then because M is bounded, M is a subset of some k-cell, call it K.  Then K is compact.  Construct an open cover of K by letting G = $\lbrace G_{x} | G_{x} = N_{\delta}(x), x \in R^{k} \rbrace$.  Because K is compact, there is a finite subcover, $\lbrace \bigcup G_{x} \rbrace$, of G over K, and because M is a subset of K, M is a subset of $\lbrace \bigcup G_{x} \rbrace$.  QED\\ 
\\
5.  Let M be a nonempty set bounded in X.  Let X be a metric space and let $\delta > 0$ be given.  Since X is a metric space, $\overline{M}$ is closed in X.  Consider M'.  Let x $\in$ M'.  Then d(x,m) $<$ r where m $\neq$ x, m $\in$ M, and m $\in N_{r}(x)$ for all r $>$ 0.  Since M is bounded in X, there exists some N such that d(p, x) $\leq$ N for all p $\in$ M.  Thus d(p,m) $\leq$ d(p,x) + d(x,m) = N + r.  Therefore $\overline{M}$ is also bounded in X.  Since $\overline{M}$ is closed and bounded, it is compact in X.  Construct an open cover of $\overline{M}$, G = $\lbrace G_{x} | G_{x} = N_{\delta}(x), x \in R^{k} \rbrace$.  Then because $\overline{M}$ is compact, and M $\subseteq \overline{M}$, there is a finite subcover of G over $\overline{M}$ which also covers M.  QED   \\
\\
6.  Let K be some open subset of \textbf{R}.  Consider G, a collection of disjoint open sets, g, in \textbf{R}.  Suppose $\bigcup_{g \in G} g = K$.  Then $K^{c} = \bigcap_{g \in G} g^{c}$.  \\
\\
7.  Let A and B be nonempty, disjoint, open sets.  Assume for contradiction that A $\bigcup$ B is connected.  Then either A $\bigcap \overline{B}$ is nonempty or B $\bigcap \overline{A}$ is nonempty.  WLOG, consider x $\in A \bigcap \overline{B}$.  Then x $\in$ A and x $\in \overline{B}$.  Since A is open, there is some neighborhood, $N_{r}(x) \subset A$ (r$>$0).  Since A and B are disjoint, this neighborhood contains no points of B.  Therefore x $\notin$ B' and since x $\notin$ B, x $\notin \overline{B}$.  Thus x $\notin$ A $\bigcap \overline{B}$.  $=><=$  So A and B must not be connected.  QED\\
\\
8.  Let K be compact in X where X is a metric space.  Let n $\in$ \textbf{$Z^{+}$} be given.  Let $G_{n} = \lbrace N_{r}(x) | g \in X, r = 1/n \rbrace$.  Then any $G_{n}$ is an open cover of K, because if k $\in$ K, k $\in$ X, so any $N_{r}(k) \in G_{n}$ contains k, for any n.  Since K is compact, there is a finite subcover of $G_{n}$ over K.  QED \\
\\
9.  Let A and B be nonempty sets in an ordered space, S.  Let x = sup A and y = inf B, and let a$\leq$b for all a$\in$A and all b$\in$B.  Suppose for contradiction that x$>$y.  WLOG, consider x.  Since x$>$y and y$\leq b \in B$, there are two cases:  $y < x \leq b $ or $y \leq b \leq x$ ($x \neq y$).  Let $y < x \leq b$ (for all $b \in B$).  Then by definition of infimum, inf B $=$ x.  This contradicts the assumption that inf B $=$ y, y $\neq$ x.  Now consider y $\leq b \leq x$.  Then there are two cases, b$<$x for all b$\in$B, or $c \leq x \leq$d for some arbitrary c and d in B.  Let b$<$x for all b in B.  Then a$\leq b < x$ for all a in A and all b in B, so by definition of supremum, sup A is some b.  This contradicts the assumption that sup A $=$ x.  Now consider $c \leq x \leq$d.  Then because inf B = y, y $\leq$ c.  Since y $\neq$ x, either y = c $<$ x or y $<$ c $\leq$ x.  Let y = c.  Then since c $<x$ and c $\geq$ a for all a in A, sup A = c = y, which contradicts the assumption that sup A = x, x $\neq$ y.  Now let y $<$ c $\leq$ x.  Then sup A = c, which contradicts the assumption that sup A = x.  Now that I've dispensed with the last of these pesky cases, I have proved that sup A $\leq$ inf B. QED  \\ 
\end{document}